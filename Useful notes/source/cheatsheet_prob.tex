%!TEX TS-program = xelatex
\documentclass[12pt, a4paper, oneside]{article}

% Можно вставить разную преамбулу
% пакеты для математики
\usepackage{amsmath,amsfonts,amssymb,amsthm,mathtools}  
\mathtoolsset{showonlyrefs=true}  % Показывать номера только у тех формул, на которые есть \eqref{} в тексте.

\usepackage[british,russian]{babel} % выбор языка для документа
\usepackage[utf8]{inputenc}          % utf8 кодировка

% Основные шрифты 
\usepackage{fontspec}         
\setmainfont{Linux Libertine O}  % задаёт основной шрифт документа

% Математические шрифты 
\usepackage{unicode-math}     
\setmathfont[math-style=upright]{Neo Euler} 

\setmathfont[range={\mathbb, \mathop}]{Asana-Math.otf}

%%%%%%%%%% Работа с картинками и таблицами %%%%%%%%%%
\usepackage{graphicx} % Для вставки рисунков                
\usepackage{graphics}
\graphicspath{{images/}{pictures/}}   % папки с картинками

\usepackage[figurename=Картинка]{caption}

\usepackage{wrapfig}    % обтекание рисунков и таблиц текстом

\usepackage{booktabs}   % таблицы как в годных книгах
\usepackage{tabularx}   % новые типы колонок
\usepackage{tabulary}   % и ещё новые типы колонок
\usepackage{float}      % возможность позиционировать объекты в нужном месте
\renewcommand{\arraystretch}{1.2}  % больше расстояние между строками


%%%%%%%%%% Графики и рисование %%%%%%%%%%
\usepackage{tikz, pgfplots}  % языки для графики
%\pgfplotsset{compat=1.16}

\usepackage{todonotes} % для вставки в документ заметок о том, что осталось сделать
% \todo{Здесь надо коэффициенты исправить}
% \missingfigure{Здесь будет Последний день Помпеи}
% \listoftodos --- печатает все поставленные \todo'шки


%%%%%%%%%% Внешний вид страницы %%%%%%%%%%

\usepackage[paper=a4paper, top=20mm, bottom=15mm,left=20mm,right=15mm]{geometry}
\usepackage{indentfirst}    % установка отступа в первом абзаце главы

\usepackage{setspace}
\setstretch{1.15}  % межстрочный интервал
\setlength{\parskip}{4mm}   % Расстояние между абзацами
% Разные длины в LaTeX: https://en.wikibooks.org/wiki/LaTeX/Lengths

% свешиваем пунктуацию
% теперь знаки пунктуации могут вылезать за правую границу текста, при этом текст выглядит ровнее
\usepackage{microtype}

% \flushbottom                            % Эта команда заставляет LaTeX чуть растягивать строки, чтобы получить идеально прямоугольную страницу
\righthyphenmin=2                       % Разрешение переноса двух и более символов
\widowpenalty=300                     % Небольшое наказание за вдовствующую строку (одна строка абзаца на этой странице, остальное --- на следующей)
\clubpenalty=3000                     % Приличное наказание за сиротствующую строку (омерзительно висящая одинокая строка в начале страницы)
\tolerance=10000     % Ещё какое-то наказание.

% мои цвета https://www.artlebedev.ru/colors/
\definecolor{titleblue}{rgb}{0.2,0.4,0.6} 
\definecolor{blue}{rgb}{0.2,0.4,0.6} 
%\definecolor{red}{rgb}{1,0,0.2} 
\definecolor{green}{rgb}{0, 0.6, 0}
\definecolor{purp}{rgb}{0.4,0,0.8} 

\definecolor{red}{RGB}{213,94,0}
\definecolor{yellow}{RGB}{240,228,66}


% цвета из geogebra 
\definecolor{litebrown}{rgb}{0.6,0.2,0}
\definecolor{darkbrown}{rgb}{0.75,0.75,0.75}

% Гиперссылки
\usepackage{xcolor}   % разные цвета

\usepackage{hyperref}
\hypersetup{
	unicode=true,           % позволяет использовать юникодные символы
	colorlinks=true,       	% true - цветные ссылки
	urlcolor=blue,          % цвет ссылки на url
	linkcolor=black,          % внутренние ссылки
	citecolor=green,        % на библиографию
	breaklinks              % если ссылка не умещается в одну строку, разбивать её на две части?
}

% меняю оформление секций 
\usepackage{titlesec}
\usepackage{sectsty}

% меняю цвет на синий
\sectionfont{\color{titleblue}}
\subsectionfont{\color{titleblue}}
\renewcommand{\thesection}{\arabic{section}.}


% выбрасываю нумерацию страниц и колонтитулы 
%\pagestyle{empty}

% синие круглые бульпоинты в списках itemize 
\usepackage{enumitem}

\definecolor{itemizeblue}{rgb}{0, 0.45, 0.70}

\newcommand*{\MyPoint}{\tikz \draw [baseline, fill=itemizeblue, draw=blue] circle (2.5pt);}
\renewcommand{\labelitemi}{\MyPoint}

\AddEnumerateCounter{\asbuk}{\@asbuk}{\cyrm}
\renewcommand{\theenumi}{\asbuk{enumi}}

% расстояние в списках
\setlist[itemize]{parsep=0.4em,itemsep=0em,topsep=0ex}
\setlist[enumerate]{parsep=0.4em,itemsep=0em,topsep=0ex}

% эпиграфы
\usepackage{epigraph}
\setlength\epigraphwidth{.6\textwidth}
\setlength\epigraphrule{0pt}

%%%%%%%%%% Свои команды %%%%%%%%%%

% Математические операторы первой необходимости:
\DeclareMathOperator{\sgn}{sign}
\DeclareMathOperator*{\argmin}{arg\,min}
\DeclareMathOperator*{\argmax}{arg\,max}
\DeclareMathOperator{\Cov}{Cov}
\DeclareMathOperator{\Var}{Var}
\DeclareMathOperator{\Corr}{Corr}

\DeclareMathOperator{\Pois}{Pois}
\DeclareMathOperator{\Geom}{Geom}
\DeclareMathOperator{\Exp}{Exp}

%\DeclareMathOperator{\E}{\mathbb{E}}
\DeclareMathOperator{\Med}{Med}
\DeclareMathOperator{\Mod}{Mod}
\DeclareMathOperator*{\plim}{plim}

% команды пореже
\newcommand{\const}{\mathrm{const}}  % const прямым начертанием
\newcommand{\iid}{\sim i\,i\,d\,\,}  % ну вы поняли...
\newcommand{\fr}[2]{\ensuremath{^{#1}/_{#2}}}   % особая дробь
\newcommand{\ind}[1]{\mathbbm{1}_{\{#1\}}} % Индикатор события
\newcommand{\dx}[1]{\,\mathrm{d}#1} % для интеграла: маленький отступ и прямая d

% одеваем шапки на частые штуки
\def \hb{\hat{\beta}}
\def \hs{\hat{s}}
\def \hy{\hat{y}}
\def \hY{\hat{Y}}
\def \he{\hat{\varepsilon}}
\def \hVar{\widehat{\Var}}
\def \hCorr{\widehat{\Corr}}
\def \hCov{\widehat{\Cov}}

% Греческие буквы
\def \a{\alpha}
\def \b{\beta}
\def \t{\tau}
\def \dt{\delta}
\def \e{\varepsilon}
\def \ga{\gamma}
\def \kp{\varkappa}
\def \la{\lambda}
\def \sg{\sigma}
\def \tt{\theta}
\def \Dt{\Delta}
\def \La{\Lambda}
\def \Sg{\Sigma}
\def \Tt{\Theta}
\def \Om{\Omega}
\def \om{\omega}

% Готика
\def \mA{\mathcal{A}}
\def \mB{\mathcal{B}}
\def \mC{\mathcal{C}}
\def \mE{\mathcal{E}}
\def \mF{\mathcal{F}}
\def \mH{\mathcal{H}}
\def \mL{\mathcal{L}}
\def \mN{\mathcal{N}}
\def \mU{\mathcal{U}}
\def \mV{\mathcal{V}}
\def \mW{\mathcal{W}}

% Жирные буквы
\def \mbb{\mathbb}
\def \RR{\mbb R}
\def \NN{\mbb N}
\def \ZZ{\mbb Z}
\def \PP{\mbb{P}}
\def \E{\mbb{E}}
\def \QQ{\mbb Q}

%%%%%%%%%% Теоремы %%%%%%%%%%
\theoremstyle{plain} % Это стиль по умолчанию.  Есть другие стили.
\newtheorem{theorem}{Теорема}[section]
\newtheorem{proposition}{Утверждение}[section]
\newtheorem{result}{Следствие}[theorem]
% счётчик подчиняется теоремному, нумерация идёт по главам согласованно между собой

% убирает курсив и что-то еще наверное делает ;)
\theoremstyle{definition}         
\newtheorem*{definition}{Определение}  % нумерация не идёт вообще


%%%%%%%%%% Задачки и решения %%%%%%%%%%
\usepackage{etoolbox}    % логические операторы для своих макросов
\usepackage{environ}
\newtoggle{lecture}

\newcounter{probNum}[section]  % счётчик для упражнений 
\NewEnviron{problem}[1]{%
    \refstepcounter{probNum}% увеличели номер на 1 
    {\noindent \textbf{\large \color{titleblue} Упражнение~\theprobNum~#1}  \\ \\ \BODY}
    {}%
  }

% Окружение, чтобы можно было убирать решения из pdf
\NewEnviron{sol}{%
  \iftoggle{lecture}
    {\noindent \textbf{\large Решение:} \\ \\ \BODY}
    {}%
  }
 
% выделение по тексту важных вещей
\newcommand{\indef}[1]{\textbf{ \color{green} #1}} 

\usepackage[normalem]{ulem}  % для зачекивания текста

% Если переключить в false, все solution исчезнут из pdf
\toggletrue{lecture}
%\togglefalse{lecture}



\title{\begin{center} \includegraphics[width=0.99\textwidth]{logo.png}\end{center}Теоретический минимум по теории вероятностей\footnote{Это краткая выжимка с основными определениями из теории вероятностей. Она не претендует на полноту. Частично шпаргалка основана на материале \url{https://github.com/bdemeshev/pr201/tree/master/cheat_sheet}}}
\date{ } %\today}
% \author{Ульянкин Филя, Романенко Саша}

\begin{document} % Конец преамбулы, начало файла

\maketitle

\section{Условная вероятность случайного события}

Условной вероятностью события $A$ при условии, что произошло событие $B$, называется число:

$$
\PP(A \mid B) = \frac{\PP(A \cap B)}{\PP(B)}.
$$

Из формулы условной вероятности можно получить формулу для вероятности произведения нескольких событий: 

$$
\PP(A \cap B) = \PP(A \mid B) \cdot \PP(B).
$$

Если событий несколько, формулу можно продолжить:

$$
\PP(A \cap B \cap C) = \PP(A \mid B, C) \cdot \PP(B \mid C) \cdot \PP(C).
$$

\newpage

\section{Независимость событий}

Говорят, что два события попарно \indef{независимы,} если верно следующее:

$$
\PP(A \cap B) = \PP(A) \cdot \PP(B)
$$

Говорят, что $n$ случайных событий \indef{независимы в совокупности,} если для любого $1 \leq k \leq n$ и любого набора различных меж собой индексов $1 \leq i_1, ..., i_k \leq n$ имеет место равенство:

$$
\PP(A_{i_1} \cap ... \cap A_{i_k}) = \PP(A_{i_1}) \cdot ... \cdot \PP(A_{i_k})
$$


\section{Формула полной вероятности}

Пусть событие $A$ происходит вместе с одним из событий $H_1, H_2, \ldots, H_k.$ Пусть эти события попарно несовместны (ещё говорят, что они составляют  \indef{полную группу.)}

Нам известны вероятности этих событий $\PP(H_1), \PP(H_2), \ldots, \PP(H_k)$, а также условные вероятности события $A$:  $\PP(A \mid H_1), \PP(A \mid H_2), \ldots, \PP(A \mid H_k)$.

Тогда вероятность события $A$ может быть вычислена по формуле:

$$
\PP(A) = \sum_{i=1}^{\infty} \PP(H_i) \cdot \PP(A \mid H_i).
$$


\section{Формула Байеса}

Пусть событие $A$ происходит вместе с одним из событий $H_1, H_2, \ldots, H_k,$ которые составляют полную группу и попарно несовместны. 

Пусть известно, что в результате испытания событие $A$ произошло. Тогда условная вероятность того, что имело место событие $H_k$, можно пересчитать по формуле:

$$
\PP(H_k \mid A) = \dfrac{\PP(H_k \cap A)}{P(A)} = \dfrac{\PP(H_k) \cdot \PP(A \mid H_k)}{\sum_{i=1}^{\infty} \PP(H_i) \cdot \PP(A \mid H_i)}
$$

\newpage

\section{Функция распределения случайной величины}

\indef{Функцией распределения} случайной величины $X$ называется функция $F_X(x),$ определённая для любого действительного числа $x \in \mathbb{R}$ и выражающая собой  вероятность того, что случайная величина $X$ примет значение, лежащее на числовой прямой левее точки $x$, то есть

$$
F_X(x) = \PP(X \leq  x).
$$

Любая функция распределения обладает следующими свойствами:

\begin{itemize}
\item Принимает значения в диапазоне от $0$ до $1$, при этом:  

$$
\lim_{x \to +\infty} F_X(x)=1 \qquad \lim_{x \to -\infty} F_X(x) = 0
$$

\item $F_X(x)$ не убывает: $F_X(x_1) \leq F_X(x_2) \quad \forall x_1 \leq x_2$

\item $F_X(x)$ непрерывна справа: $\lim\limits_{x \to x_0^+} F_X(x) = F_X(x_0)$
\end{itemize}

\section{Функция плотности случайной величины и ее свойства}

Случайная величина $X$ имеет \indef{абсолютно непрерывное распределение,} если существует неотрицательная функция $f_X(x)$ такая, что 

$$ 
F_X(x) =\int_{-\infty}^{x} f_X(t) \dx{t}.
$$

Функция $f_X(x)$ называется \indef{функцией плотности распределения} случайной величины $X$.

Свойства функции плотности:

\begin{itemize}
\item Неотрицательно определена: $ f_X(x) \geq 0$

\item Площадь под плотностью распределения всегда равна единице: $$\int_{-\infty}^{+\infty} f_X(t)  \dx{t} = 1$$

\item С помощью плотности можно найти вероятность того, что случайная величина попадёт в конкретный отрезок: 

$$
\PP(a \le X \le b) = \int_a^b f_X(x) \dx{x} = F_X(b) - F_X(a)
$$
\end{itemize}


\section{Функция совместного распределения двух случайных величин}

Пусть у нас ест две случайные величины $X$ и $Y$.  \indef{Совместной функцией распределения} двумерной случайной величины будет называться функция, определённая  $ \forall x,y \in \RR$ и выражающая собой вероятность одновременного выполнения событий $X \leq x,$ и $Y \leq y$:

$$
F(x,y) = P(X \leq x, Y \leq y)
$$

Свойства функции распределения аналогичны одномерному случаю: 

\begin{itemize}
	\item Принимает значения в диапазоне от $0$ до $1$, при этом:  
	
	$$
	F(x, -\infty) = F(-\infty, y) = F(-\infty, -\infty) = 0, \qquad  F(+\infty,+\infty) = 1
	$$
	
	\item $F_X(x)$ не убывает по каждому из своих аргументов
	
	\item $F_X(x)$ непрерывна справа по каждому из своих аргументов 
	
	\item Если один из аргументов стремится к бесконечности, то получится функция распределения другой составляющей: $$ F(x,+\infty)  = \PP(X< x, Y < +\infty) = \PP(X< x) = F_X(x) $$
\end{itemize}

Случайные величины $X$ и $Y$ \indef{независимы,} если их функция распределения равна произведению функций распределения составляющих:

$$
F(x,y) = F_{X}(x) \cdot F_{Y}(y)  
$$

Для $n$ случайных величин функцию распределения можно задать по аналогии. 

\section{Совместная плотность распределения двух случайных величин}

Случайные величины $X$, $Y$ имеют \indef{абсолютно непрерывное совместное распределение,} если существует функция $f(x, y) \ge 0$ такая, что $\forall x,y$ совместная функция распределения представима в виде: 

$$
F(x,y) = \int_{-\infty}^x \int_{-\infty}^y f(t_1, t_2) \dx{t_1}\dx{t_2}
$$

Если такая функция $f(x, y)$ существует, она называется плотностью совместного распределения случайных величин $X$ и $Y$.

Свойства совместной функции плотности:

\begin{itemize}
\item  $f(x,y) \ge 0$
\item  Площадь под совместной плотностью распределения равна единице: 
$$
\int_{-\infty}^{+\infty} \int_{-\infty}^{+\infty}  f(t_1, t_2) \dx{t_1}\dx{t_2} = 1
$$
\item Чтобы получить плотность распределения одной из составляющих, можно выинтегрировать из совместной плотности все значения другой:

\begin{equation*}
\begin{aligned}
& f_X(x) = \int_{-\infty}^{+\infty} f(x,y) \dx{y} \\ 
& f_Y(y) = \int_{-\infty}^{+\infty} f(x,y) \dx{x} \\ 
\end{aligned}
\end{equation*}
\end{itemize}

Случайные величины $X$ и $Y$ с абсолютно непрерывными распределениями \indef{независимы,} если плотность их совместного распределения существует и равна
произведению частных функций плотности:

$$
f(x,y) = f_X(x) \cdot f_Y(y)
$$

%%------------------------------------------------------
\section{Математическое ожидание}

\textbf{Интуитивно:} среднее арифметическое значение величины при многократном повторении случайного эксперимента

\indef{Математическим ожиданием} $\E(X)$ непрерывной случайной величины $X$ называется число:

$$
\E(X) = \int_{-\infty}^{+\infty} t \cdot f_X(t)  \dx{t}, 
$$ 

\indef{Математическим ожиданием} $\E(X)$ дискретной случайной величины $X$  называется число:

$$
\E(X) = \sum\limits_{k} a_k \cdot \PP(X = a_k),
$$ 

\textbf{Свойства:}

\begin{itemize}
\item $\E(a \cdot  X + b \cdot Y + c) = a \cdot \E(X) + b \cdot E(Y) + c$
\item Если $X \geq Y$ почти наверное, то $\E(X) \geq \E(Y)$
\item Если $X$ и $Y$ независимы и их математические ожидания существуют, то $$\E(X \cdot Y) = \E(X) \cdot \E(Y)$$
\end{itemize}


\section{Дисперсия}

\textbf{Интуитивно:} мера разброса случайной величины. \textbf{Геометрический смысл:} квадрат длины случайной величины.

\indef{Дисперсия} случайной величины $\Var(X)$ --- это число 

$$
\Var(X) = \E(X - \E(X))^2 =  E(X^2) - E^2(X)
$$

Дисперсия --- это среднее значение квадрата отклонения случайной величины $X$ от своего среднего.

\textbf{Свойства:}

\begin{itemize}
\item $\Var(X) \ge 0$
\item $\Var(X) = 0$ равносильно тому, что $\PP(X=\const)=1$
\item $\Var(a \cdot X + b) = a^2 \cdot \Var(X)$
\item $\Var(X + Y) = \Var(X) + \Var(Y) + 2\cdot \Cov(X,Y)$
\item $\Var(X + Y) = \Var(X) + \Var(Y)$, если величины линейно независимы
\end{itemize}

\section{Стандартное отклонение}

\indef{Стандартным отклонением} называют корень из дисперсии:

$$
\sigma(X)=\sqrt{\Var(X)}
$$

Эту величину вводят, так как дисперсия измеряется в квадратах (лет$^2$, м$^2$) и т.п.

\textbf{Свойства:}

\begin{itemize}
\item $\sigma(X) \ge 0$
\item $\sigma(X) = 0$ равносильно тому, что $\PP(X=\const)=1$
\item $\sigma(a \cdot X + b) = |a| \cdot \sigma(X)$
\end{itemize}


\section{Ковариация}

\textbf{Интуитивно:} мера линейной связи величин $X$ и $Y$

\indef{Ковариацией} между двумя случайными величинами называют величину

$$
\Cov(X, Y) = \E\left( [X - \E(X)] \cdot [Y - \E(Y)] \right) = \E(X \cdot Y) - \E(X) \cdot E(Y)
$$

\textbf{Геометрический смысл:} скалярное произведение случайных величин

\textbf{Свойства:}

\begin{itemize} 
\item Если $X$ и $Y$ независимы, то их ковариация равна нулю, но обратное неверно. Нулевая ковариация означает отсутствие линейной взаимосвязи. Взаимосвязь может быть устроена сложнее.
\item $\Cov(X, Y) = \Cov(Y, X)$
\item $\Cov(a \cdot X + b, Y) = a \cdot \Cov(X, Y)$
\item $\Cov(X + Y, Z) = \Cov(X, Z) + \Cov(Y, Z)$
\end{itemize}


\section{Корреляция}

\textbf{Интуитивно:} отнормированная мера линейной связи величин $X$ и $Y$

\indef{Коэффициентом корреляции} $\Corr(X, Y)$ случайных величин $X$ и $Y$ называется число:

$$
\Corr(X, Y) = \frac{\Cov(X, Y)}{\sqrt{\Var(X)} \cdot \sqrt{\Var(Y)}}.
$$

\textbf{Геометрический смысл:} косинус угла между случайными величинами

\textbf{Свойства:}

\begin{itemize} 
\item $ -1 \leq \Corr(X, Y) \leq 1$
\item  $\left| \Corr(X, Y) \right| = 1 \Leftrightarrow \exists a, b \in \RR: X = a \cdot Y + b$
\item $\Corr(a \cdot X + b, c \cdot Y + d) = \sgn(ac) \cdot \Corr(X, Y)$
\item $Corr(X,Y) = Corr(Y, X)$
\item $\Corr(X,Y)=0$ означает отсутствие линейной зависимости между $X$ и $Y$, но зависимость может быть устроена сложнее.
\end{itemize} 


\section{Закон больших чисел в слабой форме}

Пусть $X_1, \ldots, X_n$ попарно независимые и одинаково распределённые случайные величины с конечным вторым моментом, $\E(X_i^2) < \infty$, тогда имеет место сходимость:

$$
\frac{X_1 + \ldots + X_n}{n} \overset{p}{\to} \frac{\E(X_1) + \E(X_2) + \ldots + \E(X_n)}{n}
$$

Если у случайных величин одинаковое математическое ожидание, тогда: 

$$
\frac{X_1 + \ldots + X_n}{n} \overset{p}{\to} \E(X_1)
$$

\section{Центральная предельная теорема}

Пусть $X_1, \ldots, X_n$ случайные величины, имеющие одинаковое распределение с конечными математическим ожиданием и дисперсией:

$$
X_1, \ldots, X_n \sim iid(\mu,\sigma^2).
$$

тогда при $n \to \infty$ имеет место сходимость по распределению: 

$$
\frac{X_1 + \ldots +  X_n - \mu \cdot n}{\sqrt{n} \sigma } \overset{d}{\to} N(0,1)
$$

\section{Сходимость по вероятности}

Говорят, что последовательность случайных величин $X_1, X_2, \ldots$ сходится к случайной величине $X$ при $n \to \infty$ \indef{по вероятности,} и пишут $X_n \overset{p}{\to} X$, если для любого  $\varepsilon > 0$:

$$
\PP(|X_n - X| \ge \varepsilon) \to 0
$$

\section{Сходимость по распределению}

Говорят, что последовательность случайных величин $X_1, X_2, \ldots$ сходится к случайной величине $X$ при $n \to \infty$ \indef{по распределению,} и пишут $X_n \overset{d}{\to} X$, если $F_{X_n}(x) \to F_X(x)$ для всех $x$, в которых $F_X(x)$ непрерывна.


\section{Основные распределения}

\subsection*{Биномиальное распределение}

\indef{Биномиальное распределение} --- дискретное распределение количества успехов среди $n$ испытаний с вероятностью успеха, равной $p$. Обычно записывают как:

$$
X \sim Binom(n, p)
$$

Вероятность того, что произойдёт $k$ успехов расчитывается по формуле: 

$$
\PP(X = k) = {C}^n_k \cdot p^k \cdot (1 - p)^{n - k}, \quad k \in \{0, 1, \ldots, n\}
$$

\textbf{Пример, когда возникает:} сколько раз человек попадёт в баскетбольную корзину при $n$ бросках 

\textbf{Свойства:}

\begin{itemize} 
\item $\E(X) = n \cdot p$
\item $\Var(X) = n \cdot p \cdot (1 - p)$
\end{itemize} 


\subsection*{Распределение Пуассона}

\indef{Распределение Пуассона} --- распределение дискретной случайной величины, представляющей собой число событий, произошедших за фиксированное время, при условии, что данные события происходят с некоторой фиксированной средней интенсивностью $\lambda$ и независимо друг от друга. Хорошо подходит для моделирования счётчиков. Обычно записывают как:

$$
X \sim \Pois(\lambda)
$$

Вероятность того, что произойдёт $k$ событий расчитывается по формуле: 

$$
\PP(X = k) = e^{-\lambda} \dfrac{\lambda^k}{k!}, \quad k \in \{0, 1, \ldots, \}
$$

\textbf{Пример, когда возникает:} число лайков под фотографией, любая случайная величина счётчик, которая подчиняется аксиомам простейшего потока событий

\textbf{Свойства:}

\begin{itemize} 
\item $\E(X) = \lambda$
\item $\Var(X) = \lambda$
\end{itemize} 

\subsection*{Геометрическое распределение}

\indef{Распределение Пуассона} --- распределение дискретной случайной величины, представляющей собой номер первого успеха в серии испытаний Бернулли. Обычно записывают как:

$$
X \sim \Geom(p)
$$

Вероятность того, что номер первого успеха равен $k$ находится как:

$$
\PP(X = k) = p \cdot (1 - p)^{k - 1}
$$

\textbf{Пример, когда возникает:} номер попытки, при которой игрок попал в баскетбольную корзину

\textbf{Свойства:}

\begin{itemize} 
\item $\E(X) = \frac{1}{p}$
\item $\Var(X) = \frac{1-p}{p^2}$
\end{itemize} 


\subsection*{Равномерное распределение}

\indef{Равномерное распределение на отрезке $[a;b]$} обладает плотностью распределения: 

$$
f_X(x) =\begin{cases}
\frac{1}{b - a}, \quad x \in [a; b]  \\
0, \quad x \notin [a; b]
\end{cases}
$$

Обычно записывают как:

$$
X \sim \mU[a; b]
$$

\textbf{Пример, когда возникает:} остаток при округлении чисел

\textbf{Свойства:}

\begin{itemize} 
\item $\E(X) = \frac{a + b}{2}$
\item $\Var(X) = \frac{(b - a)^2}{12}$
\end{itemize} 


\subsection*{Экспоненциальное распределение}

\indef{Экспоненциальное распределение} обладает плотностью распределения: 

$$
f_X(x) =\begin{cases}
\lambda e^{- \lambda x}, \quad x \ge 0  \\
0, \quad x < 0
\end{cases}
$$

Обычно записывают как:

$$
X \sim \Exp(\lambda)
$$

\textbf{Пример, когда возникает:} время между событиями, имеющими распределение Пуассона (время, пока следующий человек придёт в кассу, время до следующего лайка под фото и тп)

\textbf{Свойства:}

\begin{itemize} 
\item $\E(X) = \frac{1}{\lambda}$
\item $\Var(X) = \frac{1}{\lambda^2}$
\end{itemize} 

\subsection*{Нормальное распределение}

Говорят, что у случайной величины $X$ \indef{нормальное распределение с параметрами $\mu$ и $\sigma^2$}, если она обладает плотностью распределения

$$
f_{X}(x) = \frac{1}{\sigma \sqrt{2 \pi}} e^{-\tfrac{(x - \mu)^2}{2\sigma^2}}
$$

Обычно записывают как:

$$
X \sim \mN(\mu, \sigma^2)
$$

\textbf{Пример, когда возникает:} нахождение суммы или среднего большого количества независимых одинаково распределенных величин

\textbf{Свойства:}

\begin{itemize} 
\item $\E(X) = \mu$
\item $\Var(X) = \sigma^2$
\item Если $X \sim \mN(\mu_1, \sigma_1^2)$ и $Y \sim \mN(\mu_2, \sigma_2^2)$, тогда 

$$
a\cdot X + b \cdot Y + c \sim \mN(a\cdot \mu_1 + b \cdot \mu_2 + c, a^2 \sigma_1^2 + b^2 \sigma_2^2) 
$$

\item Для нормального распределения выполняются правила одной, двух и трёх сигм: 

\begin{equation*}
\begin{aligned}
& \PP(\mu - \sigma \le X \le \mu + \sigma) \approx = 0.683 \\
& \PP(\mu - 2\cdot \sigma \le X \le \mu + 2 \cdot \sigma) \approx = 0.954 \\
& \PP(\mu - 3 \cdot \sigma \le X \le \mu + 3 \cdot \sigma) \approx = 0.997
\end{aligned}
\end{equation*}
\end{itemize} 


\subsection*{"Хи-квадрат" распределение}

Пусть случайные величины $X_1, \ldots, X_k$ независимы и одинаково распределены. Причём нормально с параметрами $0$ и $1$. Обычно такой факт записывают следующим образом: 

$$
X_1, \ldots, X_k \sim iid \hspace{2mm} N(0,1).
$$ 

Буквы $iid$ расшифровываются как identically independently distributed (независимы и одинаково распределены).

Случайная величина $Y = X_1^2 + \ldots X_k^2$ имеет \indef{распределение хи-квадрат с $k$ степенями свободы.}  Степень свободы это просто название для параметра распределения.

Обычно записывают как:

$$
Y \sim \chi^2_k
$$   

\textbf{Пример, когда возникает:} на практике тесно связано с выборочной дисперсией для нормальных выборок

\textbf{Свойства:}

\begin{itemize} 
\item $\E(\chi^2_k) = k \cdot \E(X_i^2) = k$
\item $\Var(\chi^2_k) = k \cdot \E(X_i^4) = 2k$
\item Распределение устойчиво к суммированию. То есть, если $\chi^2_k$ и $\chi^2_m$ независимы, тогда $\chi^2_k + \chi^2_m$ = $\chi^2_{k+m}$
\item $\frac{\chi^2_k}{k} \to 1$ по вероятности. 
\end{itemize} 


\subsection*{Распределение Стьюдента}

Пусть случайные величины

$$
X_0, X_1, \ldots, X_k \sim iid \hspace{2mm} N(0,1),
$$ 

тогда случайная величина $$ Y = \frac{X_0}{\sqrt{^{\chi^2_k}/_k}} $$ имеет \indef{распределение Cтьюдента с $k$ степенями свободы.}  
Обычно записывают как:

$$
Y \sim t(k)
$$   

\textbf{Пример, когда возникает:} на практике тесно связано с отношением выборочного среднего и стандартного отклонения нормальных выборок

\textbf{Свойства:}

\begin{itemize} 
\item $\E(Y) = 0, k > 1$
\item $\Var(Y) = \frac{k}{k-2}, k > 2$
\item Симметрично относительно нуля
\item $t(k) \to N(0,1)$ по распределению при $k \to \infty$
\item При $k=1$ совпадает с распределением Коши
\end{itemize} 


\subsection*{Распределение Фишера}

Говорят, что случайная величина 

$$ Y = \frac{^{\chi^2_k}/_k}{^{\chi^2_m}/_m}$$

имеет \indef{распределение Фишера c $k,m$ степенями свободы}.

Обычно записывают как:

$$
Y \sim F_{k,m}
$$   

\textbf{Пример, когда возникает:} на практике тесно связано с отношением выборочных дисперсий двух нормальных выборок

\textbf{Свойства:}

\begin{itemize} 
\item $\E(Y) = \frac{m}{m-2}, m > 2$
\item $\Var(Y) = \frac{2m^2(m + k - 2)}{k (m - 2)^2 (m - 4) }, m > 4$
\item Если $Y \sim F(k,m)$, тогда $\frac{1}{Y} \sim F(m,k)$
\item При $k \to \infty$ и $m \to \infty$ $F(k,m) \to 1$ по вероятности
\item А вот этот факт не раз всплывёт в эконометрике: $t_k^2 = F(1,k)$
\end{itemize} 

\section*{Квантильное преобразование}

\textbf{Теорема:}

Пусть функция распределения $F_X(x)$ непрерывна. Тогда случайная величина $Y = F(X)$ имеет равномерное распределение на отрезке $[0; 1]$.

\textbf{Следствие:}

Пусть $Y \sim U[0;1]$, а $F(x)$ произвольная функция распределения. Тогда случайная величина $X = F^{-1}(Y)$ будет иметь функцию распределения $F(x)$.

\end{document}