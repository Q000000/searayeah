\documentclass{article}

% math
\usepackage{amsmath}
\usepackage{amssymb}
\newcommand{\R}{\mathbb{R}}
\newcommand{\Q}{\mathbb{Q}}
\newcommand{\N}{\mathbb{N}}
\newcommand{\Z}{\mathbb{Z}}
\newcommand{\C}{\mathbb{C}}
\newcommand{\E}{\mathbb{E}}

% Russian
\usepackage[utf8]{inputenc}
\usepackage[russian]{babel}

% for double images
\usepackage{subcaption}

% Used for img import
\usepackage{import}
\usepackage{pdfpages}
\usepackage{transparent}
\usepackage{xcolor}

\newcommand{\incfig}[2][1]{%
    \def\svgwidth{#1\columnwidth}
    \import{./figures/}{#2.pdf_tex}
}

\pdfsuppresswarningpagegroup=1

\title{Сбор и анализ данных в Python}
\date{2021}
\author{Seara}
\pagenumbering{gobble}

\begin{document}
\maketitle
\newpage
\pagenumbering{arabic}

\section{Week 1}

\subsection{Основные понятия теории вероятностей}

\paragraph{Пакт} 
$X,Y,Z$ - случайные величины
\\
$x,y,z$ - какие-то конкретные значения
\\
$A,B,C$ - какие-то события
\\
$\mathbb{P}$ - вероятность
\\
$\E(X)$ - мат. ожидание
\\
$Var(X)$ - дисперсия
\\
$Cov(X,Y), \rho(X,Y)$ - ковариация и корреляция

\paragraph{Случайная величина и её распределение.}

Случайные величины бывают:
\begin{itemize}
	\item Дискретные(Множество значений конечно или счетно)
	\item Непрерывные(Принимают бесконечное, континуальное число значений)
\end{itemize}
{\bf Распределение дискретной случайной величины} - таблица, которая описывает, какие значения принимает случайная величина с какой вероятностью. Сумма вероятностей должна быть равна 1, каждая вероятность лежит между 0 и 1.
\\
{\bf Функция распределения} - функция, которая определяет вероятность события $X \leq x$, то есть
\[
F(X) = \mathbb{P}(X \leq x) = \sum \mathbb{P}(X=k)\cdot [X \leq x]
\]
\[
[X \leq x] = 
  \begin{cases}
     1 & [X \leq x] \\
     0 & otherwise
  \end{cases}
\]

\paragraph{Непрерывные случайные величины.} Распределение непрерывной случайной величины описывается плотностью распределения вероятнойтей.

\paragraph{Важные свойства.}
\begin{itemize}
	\item Плотность определениа только для непрерывных случайных величин
	\item
\end{itemize}
\end{document}
