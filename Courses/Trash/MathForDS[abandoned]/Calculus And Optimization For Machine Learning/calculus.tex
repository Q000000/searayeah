\documentclass{article}

% math
\usepackage{amsmath}
\usepackage{amssymb}
\newcommand{\R}{\mathbb{R}}
\newcommand{\Q}{\mathbb{Q}}
\newcommand{\N}{\mathbb{N}}
\newcommand{\Z}{\mathbb{Z}}
\newcommand{\C}{\mathbb{C}}

% for double images
\usepackage{subcaption}

% Used for img import
\usepackage{import}
\usepackage{pdfpages}
\usepackage{transparent}
\usepackage{xcolor}

\newcommand{\incfig}[2][1]{%
    \def\svgwidth{#1\columnwidth}
    \import{./figures/}{#2.pdf_tex}
}

\pdfsuppresswarningpagegroup=1

\title{Calculus and Optimization for Machine Learning}
\date{2021-07-27}
\author{Seara}
\pagenumbering{gobble}

\begin{document}
\maketitle
\newpage
\pagenumbering{arabic}

\section{Week 1}

\subsection{Numerical sets and mappings}

\paragraph{Set.}
A {\it set} (is commonly called non-definable and fundamental) is an entity, a {\it collection} of some objects:
\begin{itemize}
    \item An object either belongs to the set or do not
    \item One object could be included in the set only one time
    \item There is {\it no order} (even if it is trivial) on objects of the set
\end{itemize}
Sets are usually denoted by capital letters: $X, Y, A, B, R, ...$ The fact of belonging to the set is denoted as: $a \in A$.
\\
Another essential concept concerning sets is subsets. Basically, the subset is any set of (not necessarily all) elements of the given set. We consider primarily numerical sets:
\begin{itemize}
    \item Natural numbers $\N= \{1, 2, 3, 4, ...\}$
    \item Integer numbers $\Z = \{0, 1, -1, 2, -2, ...\}$
    \item Rational numbers $\Q = \{\frac{m}{n} | m \in Z, n \in N\}$
    \item Real numbers $\R=\{a_0, a_1, ...\}$
\end{itemize}

\paragraph{Mapping.}
Assume that we have two sets $X$ and $Y$. A mapping between them is, generally speaking, ordered realtion between elements of $X$ and $Y$. Consider two logic notations: quantifiers "for all" $\forall$ and "exists" $\exists$.

\paragraph{Axiomatic definition of real numbers.}
\begin{itemize}
    \item $x+y=y+x, y \cdot x = x \cdot y$ - the commutative rule
    \item $(x+y)+z = x + (y+z), (x \cdot y) \cdot z = x \cdot (x \cdot z)$ - associativity
    \item $(x+y) \cdot z = x \cdot z + y \cdot z$ - distributivity
    \item Existence of two neutral elements 1 and 0: $a \cdot 1 = a, a+0=a$
    \item Existence of the inverse elements (exept 0): $a + (-a)=0,a \cdot a^{-1}=1$
    \item Non-triviality $0 \neq 1$
    \item For any two real numbers one is able to say $a>b,a<b$ or $a=b$
    \item This order is transitive: if $a<b,b<c$, then $a<c$
    \item Completeness: Assume the we consider some section of our set into two non-intersecting sets: $ \R = A \cup B, A \cap B = \O$, such as any element $a \in A$ is smaller than any element $b \in B$. Then there is a pivot real number $c \in \R$ that $a \leq c \leq b$ for any $a$ and $b$ elements from the corresponding sets.
\end{itemize}

\paragraph{Functions.}
Consider two sets $X$ and $Y$ and mapping $f: X \mapsto Y$. This mapping (relation) called {\it functional} (or a {\it function}) if and only it associates each element of the 
$X$ set to exactly on element of the $Y$ set. $X$ set is called domain (set of arguments) and $Y$ set is called codomain (set of values).
\\
Function graph is a cetrain curve on the plane: the set of all points $(x,f(x))$ for all $x$ belonging to the function's domain.
\\
Domain - $D(f)$
\\
Codomain - $E(f)$
\\
Support - $supp(f) = \{x \in X : f(x) \neq 0\}$
\\
Composite function or composition - $g(f(x))$
\\
Vertical shift - $y = f(x) + c$
\\
Horizontal shift - $y=f(x+C)$
\\
Vertical contraction - $y=C \cdot f(x)$
\\
Horizontal contraction - $y=f(C \cdot x)$
\\
Absolute value - $y=|f(x)|$

\subsection{Limits, sequences}

\paragraph{Limit.}
Limit of the sequence - the real number that resembles our sequence the most as the element's number infinitely grows(approaches infinity). The notation: 
\[
\lim_{n \to \infty} a_n = C
\]
$\forall \varepsilon > 0: \exists N \in \N \; that \; \forall n \in \N, n \geq N \Rightarrow |a_n -C| < \varepsilon$
\\
If a sequence has the limit equal to 0, it is called {\it infinitesimal}.
Привет
\end{document} 