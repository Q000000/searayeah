\documentclass{article}
\usepackage[paper=a4paper, top=20mm, bottom=15mm,left=20mm,right=15mm]{geometry}

% math
\usepackage{amsmath}
\usepackage{amssymb}
\newcommand{\R}{\mathbb{R}}
\newcommand{\Q}{\mathbb{Q}}
\newcommand{\N}{\mathbb{N}}
\newcommand{\Z}{\mathbb{Z}}
\newcommand{\C}{\mathbb{C}}
\newcommand{\E}{\mathbb{E}}
\newcommand{\PP}{\mathbb{P}}

% Russian
\usepackage[utf8]{inputenc}
\usepackage[russian]{babel}

% for double images
\usepackage{subcaption}

% Used for img import
\usepackage{import}
\usepackage{pdfpages}
\usepackage{transparent}
\usepackage{xcolor}

\newcommand{\incfig}[2][1]{%
    \def\svgwidth{#1\columnwidth}
    \import{./figures/}{#2.pdf_tex}
}

\pdfsuppresswarningpagegroup=1

\title{Теория вероятностей}
\date{2021}
\author{Seara}
\pagenumbering{gobble}

\begin{document}
\maketitle
\newpage
\pagenumbering{arabic}

\section{Вероятность}

\subsection{Пространство элементарных событий}
$\Omega$ - {\bf пространство элементарных событий} (множество всевозможных исходов эксперимента). {\bf Случайное событие} - любое подмножество множества $\Omega$

\subsection{Классическое определение вероятности}
Если исходы опыта равновозможны, то вероятностью события $A$ называется отношение числа исходов, благоприятствующих данному событию, к числу всех возможных исходов опыта:
\[
\PP(A) = \frac{m}{n}
\]

\subsection{Условная вероятность случайного события}
Условной вероятностью события $A$ при условии, что произошло событие $B$, называется число:
\[
\PP(A \mid B) = \frac{\PP(A \cap B)}{\PP(B)}.
\]
Из формулы условной вероятности можно получить формулу для вероятности произведения нескольких событий: 
\[
\PP(A \cap B) = \PP(A \mid B) \cdot \PP(B).
\]
Если событий несколько, формулу можно продолжить:
\[
\PP(A \cap B \cap C) = \PP(A \mid B, C) \cdot \PP(B \mid C) \cdot \PP(C).
\]

\subsection{Независимость событий}
Говорят, что два события попарно {\bf независимы}, если верно следующее:
\[
\PP(A \cap B) = \PP(A) \cdot \PP(B)
\]
Говорят, что $n$ случайных событий {\bf независимы в совокупности}, если для любого $1 \leq k \leq n$ и любого набора различных меж собой индексов $1 \leq i_1, ..., i_k \leq n$ имеет место равенство:
\[
\PP(A_{i_1} \cap ... \cap A_{i_k}) = \PP(A_{i_1}) \cdot ... \cdot \PP(A_{i_k})
\]

\subsection{Формула полной вероятности}
Пусть событие $A$ происходит вместе с одним из событий $H_1, H_2, \ldots, H_k.$ Пусть эти события попарно несовместны (ещё говорят, что они составляют  {\bf полную группу)}. Нам известны вероятности этих событий $\PP(H_1), \PP(H_2), \ldots, \PP(H_k)$, а также условные вероятности события $A$:  $\PP(A \mid H_1), \PP(A \mid H_2), \ldots, \PP(A \mid H_k)$. Тогда вероятность события $A$ может быть вычислена по формуле:
\[
\PP(A) = \sum_{i=1}^{k} \PP(H_i) \cdot \PP(A \mid H_i).
\]

\subsection{Формула Байеса}
Пусть событие $A$ происходит вместе с одним из событий $H_1, H_2, \ldots, H_k,$ которые составляют полную группу и попарно несовместны. Пусть известно, что в результате испытания событие $A$ произошло. Тогда условная вероятность того, что имело место событие $H_k$, можно пересчитать по формуле:
\[
\PP(H_k \mid A) = \dfrac{\PP(H_k \cap A)}{P(A)} = \dfrac{\PP(H_k) \cdot \PP(A \mid H_k)}{\sum_{i=1}^{k} \PP(H_i) \cdot \PP(A \mid H_i)}
\]

\section{Случайная величина и её распределение}

\subsection{Типы случайных величин}
Случайные величины бывают:
\begin{itemize}
  \item Дискретные(Множество значений конечно или счетно), описывается таблицей распределения
  \item Непрерывные(Принимают бесконечное, континуальное число значений)
\end{itemize}

\subsection{Функция распределения случайной величины}
{\bf Функцией распределения} случайной величины $X$ называется функция $F(x),$ определённая для любого действительного числа $x \in \R$ и выражающая собой  вероятность того, что случайная величина $X$ примет значение, лежащее на числовой прямой левее точки $x$, то есть:
\[
F(x) = \PP(X \leq  x).
\]
\begin{itemize}
  \item Для дискретной:
  \[
    F(X) = \mathbb{P}(X \leq x) = \sum \mathbb{P}(X=k)\cdot [X \leq x]
  \]
  \[
  [X \leq x] = 
    \begin{cases}
       1 & [X \leq x] \\
       0 & otherwise
    \end{cases}
  \]
  \item Для непрерывной

\end{itemize}
\begin{figure}[h!]
  \centering
  \incfig{raspver}
\end{figure}
Любая функция распределения обладает следующими свойствами:
\begin{itemize}
  \item Принимает значения в диапазоне от $0$ до $1$, при этом:  
  \[
  \lim_{x \to +\infty} F(x)=1 \qquad \lim_{x \to -\infty} F(x) = 0
  \]
  \item $F(x)$ не убывает: $F(x_1) \leq F(x_2) \quad \forall x_1 \leq x_2$
  \item $F(x)$ непрерывна справа: $\lim\limits_{x \to x_0^+} F(x) = F(x_0)$
\end{itemize}


% \paragraph{Непрерывные случайные величины.} Распределение непрерывной случайной величины описывается плотностью распределения вероятнойтей.

% \paragraph{Важные свойства.}
% \begin{itemize}
%   \item Плотность определениа только для непрерывных случайных величин
%   \item
% \end{itemize}
\end{document}
 
